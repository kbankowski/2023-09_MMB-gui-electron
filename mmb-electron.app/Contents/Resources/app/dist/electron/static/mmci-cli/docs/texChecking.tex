\documentclass{article}
\begin{document}

\section{Adaptation of FiMod's monetary policy rule to the MMB infrastructure}

Original rule as stated in \citet{stahler2012fimod} (Eq. 76 in the
paper) relies on interest rates responding to the EA-wide inflation
gap and output growth.

\begin{equation}
\frac{R_{t}^{ecb}}{\bar{R}^{ecb}}=\left(\frac{R_{t-1}^{ecb}}{\bar{R}^{ecb}}\right)^{\rho_{R}}\left\{ \left[\left(\frac{\pi_{t}^{\tau^{c}}}{\bar{\pi}^{\tau^{c}}}\right)^{\omega}\left(\frac{\pi_{t}^{\tau^{c},*}}{\bar{\pi}^{\tau^{c},*}}\right)^{1-\omega}\right]^{\varphi_{\pi}}\left[\left(\frac{Y_{t}^{tot}}{Y_{t-1}^{\text{tot }}}\right)^{\omega}\left(\frac{Y_{t}^{tot,*}}{Y_{t-1}^{tot,*}}\right)^{1-\omega}\right]^{\varphi_{y}}\right\} ^{\left(1-\rho_{R}\right)}\exp\left(\varepsilon_{t}^{R}\right)\label{eq:originalRule}
\end{equation}
To simplify the equation I substitute the country specific variables
with EA-wide variables.

\[
\frac{R_{t}^{ecb}}{\bar{R}^{ecb}}=\left(\frac{R_{t-1}^{ecb}}{\bar{R}^{ecb}}\right)^{\rho_{R}}\left\{ \left(\frac{\pi_{t}^{\tau^{c},EA}}{\bar{\pi}^{\tau^{c},EA}}\right)^{\varphi_{\pi}}\left(\frac{Y_{t}^{tot,EA}}{Y_{t-1}^{tot,EA}}\right)^{\varphi_{y}}\right\} ^{\left(1-\rho_{R}\right)}\exp\left(\varepsilon_{t}^{R}\right)
\]
Furthermore, I take logs on both sides of the equation to make the
terms additive takes a formulation similar to the MMB.

\[
\log\left(\frac{R_{t}^{ecb}}{\bar{R}^{ecb}}\right)=\rho_{R}\log\left(\frac{R_{t-1}^{ecb}}{\bar{R}^{ecb}}\right)+\left(1-\rho_{R}\right)\left[\varphi_{\pi}\log\left(\frac{\pi_{t}^{\tau^{c},EA}}{\bar{\pi}^{\tau^{c},EA}}\right)+\varphi_{y}\log\left(\frac{Y_{t}^{tot,EA}}{Y_{t-1}^{tot,EA}}\right)\right]+\varepsilon_{t}^{R}
\]
Using approximations log terms can be written, as follows: $\log\left(\frac{R_{t}^{ecb}}{\bar{R}^{ecb}}\right)\approx\frac{R_{t}^{ecb}}{\bar{R}^{ecb}}-1=\frac{R_{t}^{ecb}-\bar{R}^{ecb}}{\bar{R}^{ecb}}$.
Furthermore, since $\bar{R}^{ecb}$ is close to unity I can define
$\frac{R_{t}^{ecb}-\bar{R}^{ecb}}{\bar{R}^{ecb}}\approx R_{t}^{ecb}-\bar{R}^{ecb}\equiv\hat{r}_{t}$,
which is the interest rate deviation. By the means of analogous transformations
and having in mind that $\bar{\pi}^{\tau^{c},EA}=1$ in the model,
inflation deviation can $\hat{\pi}_{t}^{\tau^{c},EA}$ can be obtained.

\[
\hat{r}_{t}=\rho_{R}\hat{r}_{t-1}+\left(1-\rho_{R}\right)\left[\varphi_{\pi}\hat{\pi}_{t}^{\tau^{c},EA}+\varphi_{y}\left(\log Y_{t}^{tot,EA}-\log Y_{t-1}^{tot,EA}\right)\right]+\exp\left(\varepsilon_{t}^{R}\right)
\]
Finally, the transition to a rule fully consistent with the MMB requires
that the variables, most notably interest rates and inflation are
reported in percentages in annualised terms. Against this backdrop,
I multiply the rule by the factor of $400$.

\begin{equation}
400\hat{r}_{t}=\rho_{R}400\hat{r}_{t-1}+\left(1-\rho_{R}\right)\left[\varphi_{\pi}400\hat{\pi}_{t}^{\tau^{c},EA}+4\varphi_{y}\left(100\log Y_{t}^{tot,EA}-100\log Y_{t-1}^{tot,EA}\right)\right]+400\epsilon_{t}^{R}\label{eq:mmbRule}
\end{equation}
From the rule that is fully consistent with the MMB standard specification
(see Eq. \ref{eq:mmbRule}) I can easily infer the definition of standard
MMB variables in terms of FiMod variables, used in the original rule
(Eq. \ref{eq:originalRule}).
\begin{itemize}
\item \texttt{interest}: $400\hat{r}_{t}=400\left(R_{t}^{ecb}-\bar{R}^{ecb}\right)$
\item \texttt{inflationq}: $400\hat{\pi}_{t}^{\tau^{c},EA}=400\left(\pi_{t}^{\tau^{c},EA}-1\right)$
\item \texttt{output}: $100\log Y_{t}^{tot,EA}$
\end{itemize}
Also, given Eq. \ref{eq:mmbRule} and the values of relevant model
parameters (i.e., $\rho_{R}=0.841$, $\varphi_{\pi}=1.796$, $\varphi_{y}=0.054$),
I can specify the calibration of the standard MMB monetary policy
rule, as done in Table \ref{tab:rulesParametersFiMod}.\footnote{The parameter values (i.e., $\rho_{R}=0.841$, $\varphi_{\pi}=1.796$,
$\varphi_{y}=0.054$) are taken from the model codes as distributed
by the FiMod's authors and they differ from the values specified in
the paper, which are $\rho_{R}=0.9$, $\varphi_{\pi}=1.5$, $\varphi_{y}=0.5$.
The latter set of parameters is at dome distance compared to usual,
often estimated, parameters in similar Taylor rules. Their application
results in a situation in which a standard policy shock has very profound
and long-lasting effects.}

\begin{table}[H]
\caption{Parametrisation of the FiMod specific monetary policy rule according
to the MMB setup \label{tab:rulesParametersFiMod}}

\begin{small}
\begin{center}

\begin{tabular}{l|c|c}
\hline
\textbf{} & \textbf{FiMod} & \textbf{GEAR} \\ 
\hline 
interest\_L1 & \textbf{0.841} & \textbf{0.841} \\ 
interest\_L2 & \textbf{0} & \textbf{0} \\ 
interest\_L3 & \textbf{0} & \textbf{0} \\ 
interest\_L4 & \textbf{0} & \textbf{0} \\ 
\hline 
inflationq\_ & \textbf{(1-0.841)*1.7960=0.285564} & \textbf{(1-0.841)*1.7960=0.285564} \\ 
inflationq\_L1 & \textbf{0} & \textbf{0} \\ 
inflationq\_L2 & \textbf{0} & \textbf{0} \\ 
inflationq\_L3 & \textbf{0} & \textbf{0} \\ 
inflationq\_L4 & \textbf{0} & \textbf{0} \\ 
inflationq\_F1 & \textbf{0} & \textbf{0} \\ 
inflationq\_F2 & \textbf{0} & \textbf{0} \\ 
inflationq\_F3 & \textbf{0} & \textbf{0} \\ 
inflationq\_F4 & \textbf{0} & \textbf{0} \\ 
\hline 
output\_gap\_ & \textbf{0} & \textbf{0} \\ 
output\_gap\_L1 & \textbf{0} & \textbf{0} \\ 
output\_gap\_L2 & \textbf{0} & \textbf{0} \\ 
output\_gap\_L3 & \textbf{0} & \textbf{0} \\ 
output\_gap\_L4 & \textbf{0} & \textbf{0} \\ 
output\_gap\_F1 & \textbf{0} & \textbf{0} \\ 
output\_gap\_F2 & \textbf{0} & \textbf{0} \\ 
output\_gap\_F3 & \textbf{0} & \textbf{0} \\ 
output\_gap\_F4 & \textbf{0} & \textbf{0} \\ 
\hline 
output & \textbf{(1-0.841)*0.0540*4=0.034344} & \textbf{(1-0.841)*0.0540*4=0.034344} \\ 
output\_L1 & \textbf{-(1-0.841)*0.0540*4=-0.034344} & \textbf{-(1-0.841)*0.0540*4=-0.034344} \\ 
output\_L2 & \textbf{0} & \textbf{0} \\ 
output\_L3 & \textbf{0} & \textbf{0} \\ 
output\_L4 & \textbf{0} & \textbf{0} \\ 
output\_F1 & \textbf{0} & \textbf{0} \\ 
output\_F2 & \textbf{0} & \textbf{0} \\ 
output\_F3 & \textbf{0} & \textbf{0} \\ 
output\_F4 & \textbf{0} & \textbf{0} \\ 
\hline 
std & \textbf{1.00} & \textbf{1.00} \\ 
innov & \textbf{0.25} & \textbf{0.25} \\ 
\hline 
\end{tabular}

\end{center}
\end{small}

\bigskip{}

\source{Own calculations based on the FiMod callibration.}
\end{table}

Table \ref{tab:rulesParameters} in the appendix compares parameters
of FiMod's monetary policy rule to other standard rules featuring
into MMB. FiMod's rule exhibit similarities with the CMR rule as both
have a similar persistence and encompass output growth with the difference
being that the latter operates on the basis of expected inflation
gap. Also, FiMod's rule resembles the SW rule with the latter embedding
the output gap difference rather than output growth.

The transition of the FiMod's original monetary policy rule into the
MMB set up guarantees the same results. When compared IRFs to a standard
temporary monetary policy shocks simulated in the original FiMod model
setup and in the MMB setup do not reveal any differences (see Figure
\ref{fig:FiModWithDiffSetups}).

\begin{figure}[H]
\begin{centering}
\caption{FiMod's simulation of a standard monetary policy shock in the MMB
and original model set up \label{fig:FiModWithDiffSetups}}
\par\end{centering}
\begin{centering}
\includegraphics[width=1\textwidth]{figures/FiModWithDiffSetups_short}
\par\end{centering}
\source{Own calculations with the MMB.}
\end{figure}

\end{document}